\documentclass{article}
\usepackage{amsmath}
\usepackage{amssymb}
\usepackage{hyperref}
\usepackage{listings}
\usepackage{xcolor}
\usepackage{geometry}
\usepackage{graphicx}

\geometry{a4paper, margin=1in}

\title{Verifiable Reinforcement Learning for Code and Reasoning: A Technical Overview}
\author{Repository Contributor}
\date{\today}

\begin{document}

\maketitle

\begin{abstract}
This paper presents a comprehensive framework for enhancing the reasoning and coding capabilities of Large Language Models (LLMs) through Verifiable Reinforcement Learning (RL). We address the challenge of hallucination in code generation and mathematical reasoning by replacing proxy reward models with deterministic, execution-based verification. Our system integrates Proximal Policy Optimization (PPO) and a novel Iterative Direct Preference Optimization (DPO) algorithm, enabling the model to self-improve through dynamic preference generation. We demonstrate the implementation of this framework using the Qwen2.5-Coder model, highlighting the robust reward engineering and training pipelines.
\end{abstract}

\section{Introduction}
The advent of Large Language Models (LLMs) has revolutionized code generation and automated reasoning. However, standard training methodologies, such as Supervised Fine-Tuning (SFT) on static datasets, often fail to guarantee functional correctness. Models may generate plausible-looking but incorrect code (hallucinations). Reinforcement Learning from Human Feedback (RLHF) attempts to align models with human intent, but typically relies on a learned reward model, which itself can be inaccurate or exploited (reward hacking).

To overcome these limitations, we propose a \textbf{Verifiable Reinforcement Learning} approach. By leveraging domains where ground truth is verifiable—specifically, unit tests for code and symbolic evaluation for mathematics—we provide the model with precise, non-negotiable feedback. This paper details the architecture, algorithms, and implementation of our repository, which serves as a testbed for these advanced techniques.

\section{Verifiable Reward Engineering}
The core of our framework is the replacement of stochastic reward models with deterministic verifiers.

\subsection{Code Correctness Reward}
The \texttt{CodeCorrectnessReward} module acts as the primary signal for programming tasks.
\begin{itemize}
    \item \textbf{Verification Mechanism:} For each generated code snippet, the system retrieves a corresponding "hidden" test file (template). The snippet is combined with the test harness and executed in a sandboxed subprocess.
    \item \textbf{Robustness:} The execution is protected by time-out mechanisms (to prevent infinite loops) and error handling for syntax checking.
    \item \textbf{Signal:} The reward is binary: $R = 1.0$ if all tests pass, and $R = 0.0$ otherwise. This sparse but accurate signal guides the model towards functional correctness.
\end{itemize}

\subsection{Math Correctness Reward}
For mathematical reasoning, we employ the \texttt{MathCorrectnessReward} utilizing the \texttt{SymPy} library.
\begin{itemize}
    \item \textbf{Verification:} Generated expressions are parsed and evaluated symbolically.
    \item \textbf{Validity Check:} The system ensures that the output is not only syntactically valid but also mathematically meaningful.
\end{itemize}

\section{Algorithms}
We implement two primary RL algorithms to optimize the model policy $\pi_\theta$.

\subsection{Proximal Policy Optimization (PPO)}
PPO is an on-policy gradient method used to fine-tune the model against the verifiable reward signal.
\begin{equation}
L^{CLIP}(\theta) = \hat{\mathbb{E}}_t \left[ \min(r_t(\theta)\hat{A}_t, \text{clip}(r_t(\theta), 1-\epsilon, 1+\epsilon)\hat{A}_t) \right]
\end{equation}
Our implementation (\texttt{train\_rl.py}) uses the \texttt{trl} library. We employ an adaptive KL divergence penalty to maintain the policy within a trust region of the reference model (usually the SFT checkpoint), preventing mode collapse while maximizing the expected reward.

\subsection{Iterative Direct Preference Optimization (DPO)}
Direct Preference Optimization (DPO) typically learns from a static dataset of preferences (chosen vs. rejected responses). We extend this to an \textbf{Iterative / Online} setting (\texttt{train\_dpo\_preference\_generation.py}).

\textbf{Mechanism:}
\begin{enumerate}
    \item \textbf{Generation Phase:} In each iteration $k$, the current policy $\pi_{\theta_k}$ generates multiple responses $\{y_1, ..., y_N\}$ for a given prompt $x$.
    \item \textbf{Verifiable Scoring:} The verifiers evaluate these responses.
    \item \textbf{Preference Construction:} We construct dynamic pairs $(y_w, y_l)$ where $R(y_w) > R(y_l)$. If all responses have equal scores, the sample is discarded or handled via heuristic ranking.
    \item \textbf{Optimization Phase:} The model is updated using the DPO loss:
    \begin{equation}
    L_{DPO}(\pi_\theta; \pi_{ref}) = - \mathbb{E} \left[ \log \sigma \left( \beta \log \frac{\pi_\theta(y_w|x)}{\pi_{ref}(y_w|x)} - \beta \log \frac{\pi_\theta(y_l|x)}{\pi_{ref}(y_l|x)} \right) \right]
    \end{equation}
    \item \textbf{Bootstrap:} The updated model $\pi_{\theta_{k+1}}$ is used for the next generation phase.
\end{enumerate}
This "Self-Play" mechanism allows the model to bootstrap its capabilities, effectively learning from its own successful explorations.

\section{System Architecture and Implementation}
The repository is structured to support the full lifecycle of LLM development:
\begin{itemize}
    \item \texttt{train\_sft.py}: Implements Supervised Fine-Tuning using QLoRA/PEFT for efficiency.
    \item \texttt{verifiers/}: Contains the core logic for \texttt{CodeVerifier} and \texttt{MathVerifier}.
    \item \texttt{rewards/}: Wraps verifiers into callable reward functions compatible with RL trainers.
    \item \texttt{train\_rl.py}: The PPO training loop, handling distributed rollout generation and optimization.
    \item \texttt{train\_dpo\_preference\_generation.py}: The iterative DPO loop, managing the cycle of generation, verification, and training.
\end{itemize}

All training scripts support Parameter-Efficient Fine-Tuning (PEFT) to allow training 7B+ parameter models on consumer hardware or limited compute resources.

\section{Conclusion}
This work demonstrates a practical implementation of Verifiable Reinforcement Learning. By grounding the model's training in executable reality, we move beyond the limitations of proxy rewards. The iterative DPO approach, in particular, offers a promising path for self-improving reasoning models. Future work will focus on expanding the verification domains to include theorem proving and more complex multi-step reasoning tasks.

\end{document}
